\documentclass[11pt,a4paper,sans]{moderncv} % Font sizes: 10, 11, or 12; paper sizes: a4paper, letterpaper, a5paper, legalpaper, executivepaper or landscape; font families: sans or roman
\usepackage{standalone}
\moderncvstyle{classic} % CV theme - options include: 'casual' (default), 'classic', 'oldstyle' and 'banking'
\moderncvcolor{black} % CV color - options include: 'blue' (default), 'orange', 'green', 'red', 'purple', 'grey' and 'black'

\usepackage{lipsum} % Used for inserting dummy 'Lorem ipsum' text into the template

\usepackage[scale=0.85]{geometry} % Reduce document margins
%\setlength{\hintscolumnwidth}{3cm} % Uncomment to change the width of the dates column
%\setlength{\makecvtitlenamewidth}{10cm} % For the 'classic' style, uncomment to adjust the width of the space allocated to your name

\usepackage[utf8]{inputenc}
\usepackage{booktabs}
\usepackage{fontawesome}
\usepackage{marvosym}
\usepackage{multibbl}
\newcommand\Colorhref[3][gray]{\href{#2}{\color{#1}#3}}


\firstname{THÉO}
\familyname{GUYARD}
\title{Postdoctoral Researcher in Optimization \\ PhD in Applied Mathematics}
\address{812 Rue Cherrier}{Montréal, QC H2L 1H4, Canada}
\email{guyard.theo@gmail.com} 
\mobile{(+33) 0630103928}
\age{French, 27 years old}

\extrainfo{
  \faLink~\href{https://theoguyard.github.io}{Website}
  ~
  \faGithub~\href{https://github.com/TheoGuyard}{Github}
  ~
  \faGoogle~\href{https://scholar.google.fr/citations?user=xwtKGaEAAAAJ}{Scholar}
  ~
  \faLinkedin~\href{https://www.linkedin.com/in/théo-guyard-529250155}{Linkedin}
}

% \photo[70pt][0.3pt]{picture} % The first bracket is the picture height, the second is %the thickness of the frame around the picture (0pt for no frame)
%\quote{Not Attention, Patience is all we need.}


\newcommand{\cvdoublecolumn}[2]{%
  \cvitem[.75em]{}{%
    \begin{minipage}[t]{\listdoubleitemcolumnwidth}#1\end{minipage}%
    \hfill%
    \begin{minipage}[t]{\listdoubleitemcolumnwidth}#2\end{minipage}%
    }%
}



\begin{document}

\makecvtitle

\vspace*{-1cm}

\section{Profile}

I am a researcher in applied mathematics with a strong interest in combinatorial optimization methods for machine learning.
Throughout my different projects, I have developed solid expertise in these fields.
Implementation aspects are always central to my contributions, as I consider practical performance and empirical evaluation as important as theoretical properties when it comes to real-world applications.
I am aiming to pursue a long-term career in the design of optimization methods, ideally combining advanced theoretical development, cutting-edge implementation tasks, and offering opportunities for continuous learning.

\section{Positions and Education}

\cventry{2025 - now}{Postdoctoral Researcher}{SCALE-AI Chair at Polytechnique Montréal}{}{Canada}
{New algorithms for combinatorial optimization problems inspired by machine learning techniques.~\\Supervised by \Colorhref{https://scholar.google.com/citations?user=qbO0xwUAAAAJ}{Thibaut Vidal}.}

\cventry{2021 - 2024}{PhD in Applied Mathematics}{INRIA and INSA Rennes}{}{France}
{Branch-and-bound algorithms for L0-regularized problems related to machine learning tasks.~\\ Supervised by \Colorhref{https://scholar.google.fr/citations?user=wO7HamwAAAAJ}{Cédric Herzet}, \Colorhref{https://scholar.google.fr/citations?user=2LSEQYcAAAAJ}{Clément Elvira} and \Colorhref{https://www.researchgate.net/profile/Ayse-Arslan-7}{Ayse-Nur Arlan}.}


\cventry{2018 - 2021}{MSc in Applied Mathematics with Research Cursus}{INSA Rennes}{}{France}
{Lectures and projects on optimization, machine learning, informatics, and statistics, among others. Collaborations with \Colorhref{https://www.inria.fr}{INRIA} and \Colorhref{https://irmar.univ-rennes.fr}{IRMAR} researchers alongside the MSc as a trainee.}

\cventry{2016 - 2018}{BSc in Engineering}{INSA Rennes}{}{France}
{Lectures and projects on mathematics, engineering, informatics, physics, and biology, among others.}

\section{Expertise}

\cvitem{Combinatorial Optimization}{Branch-and-bound algorithms • Structure-exploiting branching rules • Polyhedral decomposition techniques • Convexification via perspective transformations • Big-M tightening • Metaheuristics and dynamic programming for vehicle routing • Fast linear algebra operations}

\cvitem{Linear/Convex Optimization}{Active-set, proximal, primal-dual, and coordinate-descent methods • Dimensionality reduction techniques • Lagrange and Fenchel duality theory • Convex analysis theory • KKT optimality conditions • Preconditioning techniques}
  
\cvitem{Mathematical Modelling}{MIP/MINLP/SOCP/LP formulation • Model strengthening strategies • Solver libraries with integrated callbacks (Gurobi, Mosek, CPLEX, SCIP, ...) • Generic mathematical modelling libraries (\texttt{cvxpy}, \texttt{jump}, \texttt{pyomo}, \texttt{pybnb}, \texttt{ampl}, ...)}

\cvitem{Machine Learning}{Decision tree models • Generalized linear models • Nearest neighbor search methods • Statistical learning • Sparse signal processing • Fairness and privacy}

\section{Technical Skills}

\cvitem{Programming}{Python • Julia • C++ • Rust • R • Matlab}

\cvitem{Informatics}{HPC clusters with SLURM • Git • CI pipelines • Linux • LaTeX • BLAS and LAPACK}

\cvitem{Languages}{French (native) • English (fluent) • Spanish (intermediate)}

\vfill

\begin{center}
  The rest of this document details my research activities.
\end{center}
 

\clearpage

\section{Main Research Projects}

\cventry{Since 2025}{Decision Trees for Exact Combinatorial Optimization}{}{}{}
{We develop a novel kind of exact solution method for combinatorial optimization problems by encoding the feasible set into a decision tree structure. The tree only needs to be constructed once for a given feasible set and can then be used to recover the solution corresponding to any objective function via a very fast traversal operation. In contrast to traditional approaches, this method allows for very efficient re-optimization capabilities and is particularly relevant in dynamic or real-time contexts. \\ Collaborators: \Colorhref{https://scholar.google.com/citations?user=qbO0xwUAAAAJ}{Thibaut Vidal}, \Colorhref{https://scholar.google.fr/citations?user=umGuS18AAAAJ}{Maximilian Schiffer}, \Colorhref{https://scholar.google.fr/citations?user=ZHLJCRkAAAAJ}{Eduardo Uchoa}, \Colorhref{https://github.com/cleberoli}{Cleber Oliveira}.}


\cventry{Since 2025}{Routing Problems for Orbital Debris Remediation with NASA}{}{}{}
{NASA is planning to remediate debris orbiting Earth to prevent collisions with operational satellites. The routing of debris-removing spacecraft is a key component to the mission success. In collaboration with NASA, we are developing a large-scale algorithm combining vehicle routing metaheuristics and machine learning pipelines to construct an optimal route plan of debris to be visited, mitigating fuel cost and mission time for a fleet of spacecraft. \\ Collaborators: \Colorhref{https://scholar.google.com/citations?user=qbO0xwUAAAAJ}{Thibaut Vidal}, \Colorhref{https://scholar.google.fr/citations?user=UHE0jioAAAAJ}{Mingde Yin}, \Colorhref{https://www.linkedin.com/in/grahammackintosh/}{Graham Mackintosh}, \Colorhref{https://www.linkedin.com/in/thomas-templin/}{Thomas Tremplin}.}

\cventry{Since 2021}{Combinatorial Optimization Methods for L0-norm Problems}{}{}{}
{L0-norm problems are used in machine learning, signal processing, and statistics for feature selection tasks. They are of combinatorial nature and had mostly been treated approximately. We have developed a novel framework to solve them exactly through a dedicated branch-and-bound algorithm, implementing tailored acceleration strategies. It is implemented in the el0ps toolbox, now state-of-the-art to tackle these problems both regarding the framework flexibility and the numerical efficiency, achieving orders of magnitude speedups compared to competitors. \\ Collaborators: \Colorhref{https://scholar.google.fr/citations?user=wO7HamwAAAAJ}{Cédric Herzet}, \Colorhref{https://scholar.google.fr/citations?user=2LSEQYcAAAAJ}{Clément Elvira}, \Colorhref{https://www.researchgate.net/profile/Ayse-Arslan-7}{Ayse-Nur Arlan}, \Colorhref{https://scholar.google.fr/citations?user=lVz-ocwAAAAJ}{Gilles Monnoyer}.}

\cventry{Since 2019}{Dimensionality Reduction Methods for Sparse Convex Optimization}{}{}{}
{Convex sparse optimization problems can be tackled via various first-order methods. We have developed novel techniques to accelerate these algorithms through the identification of zero and non-zero entries in the problem solution during the optimization process. Upon this identification, dimensionality reduction and convergence acceleration can be achieved. Ultimately, our method allows transforming the first-order method into a second-order one, allowing fast convergence to high-accuracy solutions. Our work mainly relies on sharp convex analysis results related to Fenchel duality theory. \\ Collaborators: \Colorhref{https://scholar.google.fr/citations?user=wO7HamwAAAAJ}{Cédric Herzet}, \Colorhref{https://scholar.google.fr/citations?user=2LSEQYcAAAAJ}{Clément Elvira}.}

\section{Awards}

\cvitem{2025}{\textbf{IVADO Digital Futures Award 1st Prize.} This prize is awarded by \Colorhref{https://ivado.ca}{IVADO} (Canadian Institute for Data Valorization) during the Digital Futures event and recognizes an AI project for its innovation or impact based on a jury of journalists. It has been obtained for my work with NASA on combinatorial optimization for spatial debris remediation. The presentation is available \Colorhref{https://www.youtube.com/watch?v=cVrvMr0oS44}{here}.}

\cvitem{2023}{\textbf{ROADEF Best Student Paper Award.} This prize is awarded by the \Colorhref{https://roadef.org}{ROADEF} (French equivalent of INFORMS) and recognizes an outstanding academic or industrial contribution based on a jury of experts. It has been obtained for my work on branch-and-bounds for L0-norm problems.}

\cvitem{2022}{\textbf{SMAI Best Student Paper Award.} This prize is awarded by the \Colorhref{https://smai.emath.fr}{SMAI} (French equivalent of SIAM) and recognizes an outstanding academic or industrial contribution based on a jury of experts. It has been obtained for my work on branch-and-bounds for L0-norm problems.}

\section{Grants}

\cvitem{2025}{\textbf{IVADO R${}^3$AI Research Grant (\textasciitilde30.000 \$US).} This grant is delivered by \Colorhref{https://ivado.ca}{IVADO} (Canadian Institute for Data Valorization) and aims to support research projects related to modern optimization and machine learning methods. It has been obtained in collaboration with \Colorhref{https://scholar.google.com/citations?user=qbO0xwUAAAAJ}{Thibaut Vidal} for our work with NASA on combinatorial optimization for spatial debris remediation.}

\cvitem{2021}{\textbf{PhD Excellence Grant from INSA Rennes (\textasciitilde80.000 \$US).} One grant is delivered each year by \Colorhref{https://www.insa-rennes.fr}{INSA Rennes} to one of its students to fund a three-year PhD program. It is awarded based on academic excellence and the relevance of the proposed research project.}

\section{Publications}

% \subsection{Journal Article(Accepted)}

\cventry{2025}{El0ps: An exact L0-regularized problem solver}{}{Théo Guyard, Clément Elvira and Cédric Herzet}{Preprint}{}

\cventry{2025}{A generic branch-and-bound algorithm for L0-regularized problems}{}{Théo Guyard, Clément Elvira and Cédric Herzet}{Submitted to \textit{Mathematical Programming Computation}, Springer}{}

\cventry{2024}{Branch-and-bound algorithms for L0-regularized problems}{}{Théo Guyard}{PhD thesis}{}

\cventry{2024}{A new branch-and-bound pruning framework for L0-regularized problems}{}{Théo Guyard, Cédric Herzet, Clément Elvira and Ayse-Nur Arslan}{In \textit{International Conference on Machine Learning (ICML)}, PMLR}{}

\cventry{2023}{Safe peeling for L0-regularized least-squares}{}{Théo Guyard, Gilles Monnoyer, Cédric Herzet and Clément Elvira}{In \textit{European Signal Processing Conference (EUSIPCO)}, IEEE}{}

\cventry{2023}{An efficient solver for L0-penalized sparse problems}{}{Théo Guyard}{In \textit{Congress of the French Operation Research Society (ROADEF)}}{}

\cventry{2022}{Node-screening tests for L0-penalized least-squares problem}{}{Théo Guyard, Cédric Herzet and Clément Elvira}{In \textit{International Conference on Acoustics, Speech, and Signal Processing (ICASSP)}, IEEE}{}

\cventry{2022}{Screen \& relax: Accelerating the resolution of elastic-net by safe identification of the solution support}{}{Théo Guyard, Cédric Herzet and Clément Elvira}{In \textit{International Conference on Acoustics, Speech, and Signal Processing (ICASSP)}, IEEE}{}

\section{Softwares}

\subsection{Projects as Maintainer}

\cvitem{\texttt{el0ps}}{Python toolbox to solve L0-norm problems via flexible and numerically efficient methods. The package is available on \Colorhref{https://github.com/TheoGuyard/El0ps}{github} and \Colorhref{https://pypi.org/project/el0ps/}{pip} under AGPL-v3 license.}

\cvitem{\texttt{exprun}}{Python toolbox to automate pipelines for reproducible experiments. The package is available on \Colorhref{https://github.com/TheoGuyard/exprun}{github} and \Colorhref{https://pypi.org/project/exprun/}{pip} under MIT license.}

\cvitem{\texttt{libsvmdata}}{Julia toolbox to easily fetch libsvm datasets. The package is available on \Colorhref{https://github.com/TheoGuyard/LIBSVMdata.jl}{github} and \Colorhref{https://juliapackages.com/packages/libsvmdata}{juliapackages} under MIT license.}

\subsection{Projects as Contributor}

\cvitem{\texttt{benchopt}}{Benchmarking suite for optimization algorithms built for simplicity, transparency,
and reproducibility. The package is available on \Colorhref{https://github.com/benchopt/benchopt}{github} and \Colorhref{https://pypi.org/project/benchopt/}{pip} under BSD-3 license.}

\section{Invited Talks}

\cvitem{International Conferences}{JOPT (2025, Canada) • ICML (2024, Austria) • EUSIPCO (2023, Finland) • SIAM Conference on Optimization (2023, USA) • ICASSP (2022, Singapore)}

\cvitem{National Conferences}{IVADO digital futures (2025, Canada) • SMAI-MODE days (2024, France) • PGMO days (2023, France) • GRETSI (2023, France) • ROADEF (2023, France) • PGMO days (2022, France) • GRETSI (2022, France) • SMAI-MODE days (2022, France)}

\cvitem{Workshops}{IVADO R${}^3$AI workshop (2025, Canada) • CIRRELT seminar (2025, Canada) • IMAG seminar (2024, France) • GdR IASIS workshop (2024, France), LN2S seminar (2024, France) • INRIA EDGE seminar (2023, France) • INRIA SODA and MIND seminar (2023, France) • IRMAR seminar (2023, France) • IRMAR seminar (2022, France) }

\cvitem{Others}{CIRRELT lecture group (2025, Canada) • University of Rennes JDR (2024, France)}

\section{Scientific Duties}

\cvitem{Journal reviews}{Journal of Optimization Theory and Applications (Springer) • Signal Processing Letters (IEEE) • Journal on Data Science (INFORMS) • Inverse Problems (IOP).}

\cvitem{Conference reviews}{European Conference on Artificial Intelligence (ECAI) 2025.}

\section{Teaching}
\cventry{2024}{UE 1-07 : Optimization and numerical methods}{CS Department}{ENSAI}{France}{}
\cventry{2023 to 2024}{DMA06-OD : Discrete optimization methods}{Math. Department}{INSA Rennes}{France}{}
\cventry{2023 to 2024}{DMA09-PARCI : Sparse models and methods}{Math. Department}{INSA Rennes}{France}{}
\cventry{2022}{MathC2+ : Masterclasses for high school students on mathematical tools}{}{}{France}{}
\cventry{2021}{STP03-ALG3 - Linear algebra}{Engineering Department}{INSA Rennes}{France}{}

\section{Students}
\cventry{2025}{Lucas Langlade}{MSc research internship}{Graduated from ENPC}{France}{}

\end{document}